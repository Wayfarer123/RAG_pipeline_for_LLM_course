\documentclass[a4paper,12pt]{article}

% убираю ворнинги
% \usepackage{silence}
% Уничтожаем ворнинги
% \WarningsOff*

\usepackage{cmap}
\usepackage{amssymb}
\usepackage{amsmath}
\usepackage{amsthm}
\usepackage[T2A]{fontenc}
\usepackage[utf8]{inputenc}
\usepackage[russian]{babel}
\usepackage{indentfirst}
\usepackage{epigraph}
\usepackage{extarrows}
\usepackage{framed}
\renewcommand{\epigraphsize}{\small}
\usepackage{relsize}
\usepackage{siunitx}
\usepackage{multicol}
\usepackage{makecell}
\usepackage{enumitem}
\usepackage{stackengine,scalerel}
\def\dclesize{\ThisStyle{\raisebox{-.7pt}{\scalebox{1.45}{$\SavedStyle\bigcirc$}}}}
\def\dcle{\ensurestackMath{\stackon[0pt]{\leqslant}{\dclesize}}}
\def\cleq{\def\stacktype{L}\mathbin{\scalerel*{\dcle}{\dclesize}}}
% \pagestyle{empty}

\usepackage[left=1.25cm,right=1.25cm,
top=1.75cm,bottom=1.75cm]{geometry}
\usepackage{graphicx}
\graphicspath{ {./images/} }
\newcommand{\RomanNumeralCaps}[1] {\MakeUppercase{\romannumeral #1}}

\usepackage{amsmath}
\usepackage[makeroom]{cancel}

\newcommand{\mysection}[2]{\setcounter{section}{#1}\addtocounter{section}{-1}\section{#2}}
\newcommand{\envalias}[2]{\newenvironment{#1}{\begin{#2}}{\end{#2}}}

% код
\DeclareRobustCommand{\svdots}{% s for `scaling'
\, \vcenter{%
\offinterlineskip
\hbox{.}
\vskip0.25\normalbaselineskip
\hbox{.}
\vskip0.25\normalbaselineskip
\hbox{.}%
}%
\,
}
\usepackage{listings}
\usepackage[unicode, pdftex]{hyperref}
\usepackage{xcolor}

\usepackage{dsfont}
\usepackage{mathabx}
\usepackage{multirow}
\usepackage{hhline}
\usepackage{float}

\usepackage{cancel}
\usepackage{breqn}

\usepackage{wrapfig}
\usepackage{stmaryrd}
\usepackage{mathrsfs}

\definecolor{linkcolor}{HTML}{50006b} % цвет ссылок
%\definecolor{urlcolor}{HTML}{107896} % цвет гиперссылок
\definecolor{urlcolor}{HTML}{50006b} % цвет гиперссылок
 
\hypersetup{pdfstartview=FitH,  linkcolor=linkcolor,urlcolor=urlcolor, colorlinks=true}

\definecolor{codegreen}{rgb}{0,0.6,0}
\definecolor{codegray}{rgb}{0.5,0.5,0.5}
\definecolor{codepurple}{rgb}{0.58,0,0.82}
\definecolor{backcolour}{cmyk}{0,0,0,0.05}

\lstdefinestyle{mystyle}{
backgroundcolor=\color{backcolour},
commentstyle=\color{codegreen},
keywordstyle=\color{magenta},
numberstyle=\tiny\color{codegray},
stringstyle=\color{codepurple},
basicstyle=\ttfamily\footnotesize,
breakatwhitespace=false,
breaklines=true,
captionpos=b,
keepspaces=true,
numbers=left,
numbersep=5pt,
showspaces=false,
showstringspaces=false,
showtabs=false,
tabsize=2,
texcl=true
}
\lstset{extendedchars=\true, style=mystyle}

\usepackage[pages = some]{background}
\backgroundsetup{
	scale = 1,
	angle = 0,
	opacity = 1,
	contents = {\includegraphics[height = \paperheight, keepaspectratio]{background.jpg}}}
	
% объявление новых макрокоманд
% \newcommand{\Def}{\textbf{Определение:} } please use \begin{definition}
% \newcommand{\Statement}{\textbf{Утверждение:} }  please use \begin{statement}
% \newcommand{\Lemma}{\textbf{Лемма:} } please use \begin{lemma}
% \newcommand{\Th}{\textbf{Теорема:} } please use \begin{theorem}
\newcommand{\Task}{\textbf{Задача:} }
\newcommand{\Solution}{\textbf{Решение:} }
\newcommand{\Example}{\textbf{Пример:} }
% \newcommand{\Note}{\textbf{Замечание:} } please use \begin{lemmanote}
% \newcommand{\Cor}{\textbf{Следствие:} } please use \begin{corollary}
\newcommand{\Vars}{\textbf{Введем обозначения:} } 
% \newcommand{\Proof}{$\blacktriangle$ } please use \begin{proof}
% \newcommand{\EndProof}{$\blacksquare$ } please use \end{proof}
\newcommand{\del}{\partial}
\newcommand{\probspace}{(\Omega,\: \F,\: P)}

\renewenvironment{leftbar}[2][\hsize]
{
    \def\FrameCommand
    {
        {\hspace{20pt}
        \color{gray}\vrule width 3pt}
        \hspace{0pt}
    }
    \MakeFramed{\hsize#1\advance\hsize-\width\FrameRestore}
}
{\endMakeFramed}

\newcommand{\norm}{\triangleleft}
\newcommand{\bnorm}{\triangleright}

\DeclareMathOperator{\sgn}{sgn}
\DeclareMathOperator{\Int}{int}
\DeclareMathOperator{\rk}{rk}
\DeclareMathOperator{\ke}{Ker}
\DeclareMathOperator{\im}{Im}
\DeclareMathOperator{\re}{Re}
\DeclareMathOperator{\cha}{char}
\DeclareMathOperator{\ord}{ord}
\DeclareMathOperator{\tr}{tr}
\DeclareMathOperator{\St}{St}
\DeclareMathOperator{\Aut}{Aut}
\DeclareMathOperator{\Inn}{Inn}
\DeclareMathOperator{\End}{End}
\DeclareMathOperator{\GL}{GL}
\DeclareMathOperator{\SL}{SL}
\DeclareMathOperator{\diag}{diag}

\newcommand{\angles}[1]{\left\langle{#1}\right\rangle}
\newcommand{\abs}[1]{\left|{#1}\right|}
\newcommand{\brackets}[1]{\left({#1}\right)}

\newcommand{\N}{\mathbb{N}}
\newcommand{\Z}{\mathbb{Z}}
\newcommand{\R}{\mathbb{R}}
\newcommand{\F}{\mathcal{F}}
\newcommand{\B}{\mathcal{B}}
\newcommand{\G}{\mathcal{G}}
\renewcommand{\C}{\mathbb{C}}
\newcommand{\Cm}{\mathbb{C}}
\DeclareMathOperator{\E}{\mathbb{E}}
\newcommand{\D} { {\mathbb{D}}}
\newcommand{\indep}{\perp \!\!\! \perp}
\newcommand{\Norm}{\mathcal{N}}
\newcommand{\uniconv}{\rightrightarrows}

\newcommand{\divby}{
	\mathrel{\vbox{\baselineskip.65ex\lineskiplimit0pt\hbox{.}\hbox{.}\hbox{.}}}
}
\newcommand{\notdivby}{\centernot\divby}
\newcommand{\Q}{\mathbb{Q}}
\newcommand{\K}{\mathbb{K}}
\newcommand{\Kk}{\mathbb{K}}
\newcommand{\id}{\mathrm{id}}
\newcommand{\imp}[2]{(#1\,\,$\ra$\,\,#2)\,\,}
\newcommand{\eqv}[2]{(#1\,\,$\lra$\,\,#2)\,\,}
\newcommand{\Chi}{\scalebox{1.1}{\raisebox{\depth}{$\chi$}}}
\newcommand{\crad}{R_{\text{сх}} }

\let\bs\backslash
\let\vect\overline
\let\normal\trianglelefteqslant
\let\lra\Leftrightarrow
\let\ra\Rightarrow
\let\la\Leftarrow
\let\gl\langle
\let\gr\rangle
\let\emb\hookrightarrow
\let\mc\mathcal
\let\mf\mathfrak

\renewenvironment{proof}{{\noindent\textbf{Доказательство.}}}{\qed}
\theoremstyle{definition}
\newtheorem{theorem}{Теорема}[section]
\newtheorem{corollary}{Следствие}[theorem]
\newtheorem{lemma}[theorem]{Лемма}
\newtheorem{statement}[theorem]{Утверждение}
\newtheorem{proposition}[theorem]{Утверждение}
\newtheorem{lemmanote}[theorem]{Замечание}
\newtheorem{note}[theorem]{Замечание}
\newtheorem{example}[theorem]{Пример}
\newtheorem{counterexample}[theorem]{Контрпример}
\newtheorem{trait}[theorem]{Свойство}
\newtheorem{definition}[theorem]{Определение}
\newcommand{\blank}[1]{\hspace*{#1}}

%%% Графика
\usepackage{tikz}        % Графический пакет tikz
\usepackage{tikz-cd}     % Коммутативные диаграммы
\usepackage{tkz-euclide} % Геометрия
\usepackage{stackengine} % Многострочные тексты в картинках
\usetikzlibrary{angles, babel, quotes, fadings, shapes.geometric}
\tikzfading[name=fade out, inner color=transparent!0, outer color=transparent!100]

\makeatletter
\renewcommand{\dddot}[1]{%
  {\mathop{\kern\z@#1}\limits^{\vbox to-1.4\ex@{\kern-\tw@\ex@
   \hbox{\normalfont ...}\vss}}}}
\renewcommand{\ddddot}[1]{%
  {\mathop{\kern\z@#1}\limits^{\vbox to-1.4\ex@{\kern-\tw@\ex@
   \hbox{\normalfont....}\vss}}}}
\makeatother

\newcommand{\charac}[2]{\varphi_{#1}\brackets{#2}} % Хар.функция от распределения с.в. от аргумента
\newcommand{\eps}{\varepsilon}

\newcommand*{\hm}[1]{#1\nobreak\discretionary{}{\hbox{$\mathsurround=0pt #1$}}{}}

\let\bs\backslash
\let\vect\overline
\let\normal\trianglelefteqslant
\let\lra\Leftrightarrow
\let\ra\Rightarrow
\let\la\Leftarrow
\let\gl\langle
\let\gr\rangle
\let\sd\leftthreetimes
\let\emb\hookrightarrow
\let\convu\rightrightarrows

\newcommand{\CM}{\overline{\mathbb{C}}}
\newcommand{\varnormal}{\mathbin{\reflectbox{$\normal$}}}
\newcommand{\System}[1]{
	\left\{\begin{aligned}#1\end{aligned}\right.
}
\newcommand{\Root}[2]{
	\left\{\!\!\sqrt[#1]{#2}\right\}
}
\newcommand{\convlr}{\rightarrow_{\text{\tiny ЛР}}}

\renewcommand\labelitemi{$\triangleright$}

\renewcommand{\epsilon}{\ensuremath{\varepsilon}}
\renewcommand{\phi}{\ensuremath{\varphi}}
\renewcommand{\kappa}{\ensuremath{\varkappa}}
\renewcommand{\le}{\ensuremath{\leqslant}}
\renewcommand{\leq}{\ensuremath{\leqslant}}
\renewcommand{\ge}{\ensuremath{\geqslant}}
\renewcommand{\geq}{\ensuremath{\geqslant}}
\renewcommand{\emptyset}{\ensuremath{\varnothing}}

\DeclareMathOperator{\Arg}{Arg}
\DeclareMathOperator{\Ln}{Ln}

\DeclareMathOperator*{\res}{res}

\usepackage{framed}
\usepackage{xcolor}
\usepackage{lipsum}% dummy text


\newlength{\leftbarwidth}
\setlength{\leftbarwidth}{3pt}
\newlength{\leftbarsep}
\setlength{\leftbarsep}{10pt}

\newcommand*{\leftbarcolorcmd}{\color{leftbarcolor}}% as a command to be more flexible
\colorlet{leftbarcolor}{black}

\renewenvironment{leftbar}{%
    \def\FrameCommand{{\leftbarcolorcmd{\vrule width \leftbarwidth\relax\hspace {\leftbarsep}}}}%
    \MakeFramed {\advance \hsize -\width \FrameRestore }%
}{%
    \endMakeFramed
}

\definecolor{crimson}{HTML}{DC143C}
\definecolor{teal}{HTML}{008080}

\colorlet{leftbarcolor}{gray}
\envalias{unnecessary}{leftbar}

\newcommand{\argmin}{\text{argmin}}